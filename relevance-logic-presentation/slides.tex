\documentclass[14pt]{beamer}
\usepackage[english, russian]{babel}
\usepackage[utf8x]{inputenc}
\usepackage{itmobeamer}

\title[Релевантная логика]{Релевантная логика}
\author[]{Симоненко Е.А., <easimonenko@mail.ru>}
\institute[]{Университет ИТМО}
\date[]{Санкт-Петербург, 2018}

\begin{document}

\begin{darkbars}
    \begin{frame}[noheader,nologo,noframenumbering]
        \titlepage
    \end{frame}
\end{darkbars}

\begin{frame}
	\frametitle{Содержание}
	\tableofcontents
\end{frame}

\section{Мотивация}

\begin{frame}[nologo]{Мотивация}
	\begin{itemize}
		\item ``relevant'' -- относящийся к делу
		\item попытка избежать парадоксов материальной и строгой импликации
		
	\end{itemize}
\end{frame}

\begin{frame}[nologo]{Мотивация}
	\begin{itemize}
		\item Материальная импликация (классическая логика и булева алгебра): 
		\[ A \rightarrow B \equiv \neg A \vee B \]
		\item Строгая импликация (модальная логика): \[ A \rightarrow B \equiv 
		\forall w~\neg A(w) \vee B(w) \]
	\end{itemize}
\end{frame}

\begin{frame}[nologo]{Мотивация}
	Парадоксы материальной импликации:
	
	\begin{itemize}
		\item $ p \rightarrow (q \rightarrow p) $
		\item $ \neg p \rightarrow (p \rightarrow q) $
		\item $ (p \rightarrow q) \vee (q \rightarrow p) $
	\end{itemize}
\end{frame}

\begin{frame}[nologo]{Мотивация}
	Парадоксы строгой импликации:
	
	\begin{itemize}
		\item $ (p \wedge \neg p) \rightarrow q $
		\item $ p \rightarrow (q \rightarrow q) $
		\item $ p \rightarrow (q \vee \neg q) $
	\end{itemize}
\end{frame}

\begin{frame}[nologo]{Мотивация}
	Hugh McColl, 1908
	
	Формулы противоречат интуиции: если A, то B.
	
	Посылка никак не относится к заключению.
\end{frame}

\section{Основная идея}

\begin{frame}[nologo]{Основная идея}
	\begin{itemize}
		\item В парадоксах неправильно то, что посылка и заключение затрагивают 
		совершенно различные темы.
		\item Принцип общих переменных: никакая формула вида $ A \rightarrow B 
		$ не может быть доказана, если формулы $ A $ и $ B $ не имеют общих 
		пропозициональных переменных; никакое умозаключение не может быть 
		истинным, если посылка и заключение не имеют хотя бы одной общей 
		пропозициональной переменной.
	\end{itemize}
\end{frame}

\section{Релевантные логические системы}

\begin{frame}[nologo]{Релевантные логические системы}
\begin{itemize}
	\item Логика E релевантного следования (Андерсон, Белнап).
	\item Система R релевантной импликации (Андерсон, Белнап).
	\item Логика NR на базе логики R с оператором необходимости (Мейер).
	\item Логики NR и E имеют существенные различия (Максимова).
	\item Слабая система S (Мейер, Мартин).
\end{itemize}
\end{frame}

\begin{frame}[nologo]{Релевантные логические системы}
Среди аргументов в пользу слабых систем то, что, в отличие от R или E, многие 
из них разрешимы. Другое свойство слабых систем, которое делает их 
привлекательными,-- то, что они могут быть использованы для построения наивной 
теории множеств. Наивная теория множеств -- это теория множеств, которая 
включает в себя аксиому свертывания, т.е. для любой формулы $A(y)$, $\exists x 
\forall y (y \in x \Leftrightarrow A(y))$.
\end{frame}

\subsection{Логика S}

\begin{frame}[nologo]{Логика S}
Язык:
\begin{itemize}
	\item пропозициональные переменные
	\item скобки
	\item связка (импликация)
\end{itemize}
\end{frame}

\begin{frame}[nologo]{Логика S}
Аксиомы:
\begin{itemize}
	\item префиксация: $(B \rightarrow C) \rightarrow ((A \rightarrow B) 
	\rightarrow (A \rightarrow C))$
	\item суффиксация: $(A \rightarrow B) \rightarrow ((B \rightarrow C) 
	\rightarrow (A \rightarrow C))$
\end{itemize}
\end{frame}

\begin{frame}[nologo]{Логика S}
Правила:
\begin{itemize}
	\item транзитивность: $A \rightarrow B, B \rightarrow C \vdash A 
	\rightarrow C$
	\item суффиксация: $A \rightarrow B \vdash (B \rightarrow C) \rightarrow (A 
	\rightarrow C)$
	\item префиксация: $B \rightarrow C \vdash (A \rightarrow B) \rightarrow (A 
	\rightarrow C)$
\end{itemize}
\end{frame}

\subsection{Логика T-W}

\begin{frame}[nologo]{Логика T-W}
Логика T-W является логикой S с дополнительной аксиомой identity: $$A 
\rightarrow A$$.

Мартин доказал, что аксиома identity не является теоремой в логике S. Это 
является следствием того, что по теореме Мартина, если $A \rightarrow B$ и $B 
\rightarrow A$ доказуемы, то $A$ и $B$ являются одной и той же формулой.
\end{frame}

\section{Семантика}

\begin{frame}[nologo]{Семантика}
Семантика тернарного отношения Рутли и Мейера (Richard Routley и Robert K. 
Meyer).

``Тернарное'' - значит отношение имеет три параметра, например: X ударил 
предметом Y по Z.

Эта семантика - развитие "семантики полурешеток" Аласдаира Уркухарта (Alasdair 
Urquhart) (Urquhart 1972).
\end{frame}

\begin{frame}[nologo]{Семантика}
Как и семантика модальной логики, семантика релевантной логики связывает 
отношение истинности с мирами. Но Рутли и Мейер сделали модальную логику чуть 
лучше, используя трехместное отношение между мирами. Это допускает миры, в 
которых нельзя доказать $ q \rightarrow q $ и, как следствие, миры, где нельзя 
доказать $ p \rightarrow (q \rightarrow q) $.
\end{frame}

\begin{frame}[nologo]{Семантика}
Условие истинности для импликации таково:

$ B \rightarrow C $ истинно в мире $ a $ тогда и только тогда, когда для всех 
миров $ b $ и $ c $, таких, что $ Rabc $ ($ R $ - отношение возможности) $ B $ 
ложно в $ b $ или $ C $ истинно в $ c $.
\end{frame}

\section{Интерпретация}

\begin{frame}[nologo]{Интерпретация}
За последнее время было разработано три интерпретации, основанных на теориях, 
описывающих природу информации:
\begin{itemize}
	\item интерпретация Данна
	\item интерпретация Барвиса и Ресталла
	\item интерпретация Мареса
\end{itemize}
\end{frame}

\begin{frame}[nologo]{Интерпретация}
Интерпретация тернарного отношения, принадлежащая \emph{Данну}, продолжает идеи 
Уркухартовой семантики полурешёток. В семантике Уркухарта, вместо того, чтобы 
трактовать значения переменных как возможные (или невозможные) миры, они 
рассматриваются как фрагменты информации. В семантике полурешёток оператор $ 
"o" $ принимает два операнда, и формула $ a~o~b $ означает комбинацию 
информаций в $ a $ и $ b $.
\end{frame}

\begin{frame}[nologo]{Интерпретация}
Семантика Рутли-Мейера не содержит никакой операции ``комбинирования'' миров, 
но мы можем получить примерно такой же результат с помощью тернарного 
отношения. В понимании Данна, $ Rabc $ означает, что ``комбинация 
информационных состояний $ a $ и $ b $ содержится в информационном состоянии $ 
c $.'' (Dunn 1986).
\end{frame}

\begin{frame}[nologo]{Интерпретация}
Интерпретация Джона Барвиса (Jon Barwise) (1993) и Ресталла (Restall) (1996):

С этой точки зрения миры можно представить как информационно-теоретические 
``сайты'' или ``каналы''. Сайт - это контекст, в котором получена информация, а 
канал -- это средство, через которое получена информация.
\end{frame}

\begin{frame}[nologo]{Интерпретация}
Применяя теорию каналов для интерпретации семантики Рутли-Мейера, 
мы считаем, что $ Rabc $ имеет следующий смысл: $ a $ -- это 
информационно-теоретический канал между сайтами $ b $ и $ c $. Тогда мы 
полагаем, что $ B \rightarrow C $ истинно в $ a $ -- это значит, что всякий 
раз, когда $ a $ соединяет сайт $ b $, на котором получают $ B $, с сайтом $ c 
$, то на сайте $ c $ получают $ C $.
\end{frame}

\begin{frame}[nologo]{Интерпретация}
Интерпретация Мареса (Mares) (1997):

Используется теория информации по Дэвиду Израэлю и Джону Перри (Israel and John 
Perry (1990)). Согласно их теории, помимо другой информации мир содержит 
информационные связи такие, как законы природы, обычаи и т.д. С этой точки 
зрения $ Rabc $ тогда и только тогда, когда, согласно информационным связям 
мира $ a $, вся информаций, которую несет несет мир $ b $, содержится в $ c $.
\end{frame}

\begin{frame}[nologo]{Интерпретация}
Например, Ньютонов мир содержит информацию о том, что любая материя притягивает 
другую материю. В терминах этой теории информации этот мир содержит информацию 
о том, что две материальные вещи несут информацию о том, что они притягивают 
друг друга. Таким образом, например, если $ a $ -- Ньютонов мир, и 
информация о том, что $ x $ и $ y $ материальны, содержится в $ 
b $, тогда информация о том, что $ x $ и $ y $ притягивают друг друга, 
содержится в $ c $.
\end{frame}

\section{Семантика. Продолжение}

\begin{frame}[nologo]{Семантика. Продолжение}
Самого со себе использования тернарного отношения не достаточно чтобы избежать 
парадоксов импликации. Из всего, о чем мы говорили до сих пор, не очевидно, как 
семантика позволяет избежать парадоксов вроде $ (p \wedge \neg p) \rightarrow q 
$ и $ p \rightarrow (q \vee \neg q) $. Эти парадоксы избегаются через включение 
\emph{противоречивых} и \emph{недвузначных} миров в семантику.

Под противоречивым миром понимается мир, где не действует закон противоречия $ 
p \wedge \neg p \equiv F $, а под недвузначным -- мир, где не действует 
закон исключенного третьего $ p \vee \neg p \equiv T $.
\end{frame}

\begin{frame}[nologo]{Семантика. Продолжение}
Например, если невозможны миры, в которых истинно $ (p \wedge \neg p) $, то, 
согласно нашему условию истинности для операции $\rightarrow$, формула $(p 
\wedge \neg p) \rightarrow q$ также будет истинной всюду. Аналогично, если во 
всех мирах истинно $(q \vee \neg q)$, то во всех мирах истинно $ p \rightarrow 
(q \vee \neg q)$.
\end{frame}

\begin{frame}[nologo]{Семантика. Продолжение}
Это приводит нас к семантике для операции отрицания. Использование недвузначных 
и противоречивых миров требует неклассического условия истинности для 
отрицания. В начале 70-х, Ричард и Вал Рутли (Richard и Val Routley) изобрели 
их ``оператор звездочку'' для трактовки отрицания. Этот оператор является 
оператором над мирами. Для каждого мира $a$ существует мир $a*$. И $\neg A$ 
истинно в $a$ тогда и только тогда, когда $A$ ложно в $a*$.
\end{frame}

\begin{frame}[nologo]{Семантика. Продолжение}
И снова у нас возникают трудности с интерпретацией части формальной семантики. 
Одна интерпретация звездочки Рутли принадлежит Данну (1993). Данн использовал 
бинарное отношение $C$ для миров. $Cab$ означает, что $b$ совместимо с $a$. 
Тогда $a*$ - это максимальный мир (т.е. мир, содержащий максимум информации), 
из тех, что совместимы с $a$.
\end{frame}

\section{Теория доказательств}

\begin{frame}[nologo]{Теория доказательств}
В настоящее время существует множество подходов к теории доказательств для 
релевантной логики. Тут и последовательные вычисления по Грегори Минтсу и Данну 
(Gregory Mints (1972) и J.M. Dunn (1973)) для фрагмента (без отрицания) логики 
R и элегантный и очень общий подход, названный "Display Logic" Нюэля Белнапа 
(Nuel Belnap) (1982).
\end{frame}

\begin{frame}[nologo]{Теория доказательств}
Система естественного вывода Андерсона и Белнапа для релевантной логики R 
основана на системах естественного вывода Фитча (Fitch) для классической и 
интуиционистской логики.
\end{frame}

\section{Ссылки}

\begin{frame}[nologo]{Ссылки}
\begin{itemize}
	\item \url{http://psi-logic.narod.ru/psi/rele.htm}
	\item \url{https://plato.stanford.edu/entries/logic-relevance/index.html}
	\item \url{https://plato.stanford.edu/entries/logic-relevance/logics.html}
	\item \url{https://en.wikipedia.org/wiki/Relevance_logic}
	\item Сидоренко Е.А. Релевантная логика. -- М.: 2000. -- 243 с.
\end{itemize}
\end{frame}

\itmothankyou

\end{document}

